\documentclass[12pt,a4paper]{article}

%%%%%%%%%%%%%%%%%%%%%%%%% packages %%%%%%%%%%%%%%%%%%%%%%%%
\usepackage{graphicx}
\usepackage{tabulary}

\usepackage{amsmath}
\usepackage{fancyhdr}
\usepackage{amssymb}
\usepackage{amsthm}
\usepackage{placeins}
\usepackage{amsfonts}
\usepackage{graphicx}
\usepackage[all]{xy}
\usepackage{tikz}
\usepackage{verbatim}
\usepackage[left=2cm,right=2cm,top=3cm,bottom=2.5cm]{geometry}
\usepackage{hyperref}
\usepackage{caption}
\usepackage{subcaption}
\usepackage{multirow}
\usepackage{psfrag}
\usepackage{minted}
\usepackage[T1]{fontenc}


\graphicspath{ {./images/} }


\newcommand{\course}{\textbf{Introdução ao Aprendizado de Máquina}}


\newtheorem{thm}{Theorem}
\newtheorem{lem}[thm]{Lemma}
\newtheorem{defn}[thm]{Definition}
\newtheorem{exa}[thm]{Example}
\newtheorem{rem}[thm]{Remark}
\newtheorem{coro}[thm]{Corollary}
\newtheorem{quest}{Question}[section]


\usepackage{fancyhdr}
\pagestyle{fancy}
\rhead{ \thepage}
\renewcommand{\headrulewidth}{0.4pt}
\renewcommand{\footrulewidth}{0.4pt}


\begin{document}

%%%%%%%%%%%%%%%%%%%%%%% title page %%%%%%%%%%%%%%%%%%%%%%%%%%
\thispagestyle{empty}
\begin{center}
	%\textbf{AFRICAN INSTITUTE FOR MATHEMATICAL SCIENCES \\[0.5cm]
	%(AIMS RWANDA, KIGALI)}
	\vspace{0.5cm}
\end{center}
%%%%%%%%%%%%%%%%%%%%% assignment information %%%%%%%%%%%%%%%%
\noindent
\rule{17cm}{0.2cm}\\[0.3cm]

Disciplina: \course \hfill Date: \today\\
\rule{17cm}{0.05cm}

\section{Integrantes}

Daniel Angelo Esteves Lawand -  \#USP 10297693\\
Thiago Gomes Verissimo - \#USP 5385361\\
Victor Luiz Serra - \#USP 10362650\\

\section{Objetivo}

O projeto final do grupo tem como objetivo desenvolver um modelo de aprendizado supervisionado em regressão, a fim de prever, dado um determinado conjunto de features, a nota do ENEM de um aluno em específico. Essa nota será composta pela média do aluno nas áreas de Nota da prova de Ciências da Natureza, Ciências Humanas, Linguagens e Códigos, Matemática, excluindo a nota de redação.

\section{Dados a serem usados}
Para o treinamento do modelo iremos utilizar a base de dados do ENEM[1]. Nela temos informações de cada inscrito que prestou a prova do ENEM no ano de 2020, incluindo dados à respeito do aluno e da sua escola. Juntamente a ela utilizaremos duas outras bases de dados.São estas a base de dados da SAEB[2] (Sistema de Avaliação da Educação Básica) e a do Censo Escolar[3]. As utilizaremos à fim de complementar as informações e nos fornecer uma visão da qualidade do ensino no município que o participante se encontra, levando em consideração o âmbito administrativo e a localidade (rural ou urbana) em que a escola do participante se insere.

\section{Bibliografia}
\renewcommand{\section}[2]{}%
\begin{thebibliography}{3}
\bibitem{enem}
INEP (2021), Microdados do Enem 2020. Disponível em: https://www.gov.br/inep/pt-br/acesso-a-informacao/dados-abertos/microdados/enem

\bibitem{saeb}
INEP (2019) Sistema de Avaliação da Educação Básica (SAEB). Disponível em: https://www.gov.br/inep/pt-br/areas-de-atuacao/avaliacao-e-exames-educacionais/saeb/resultados

\bibitem{censo}
INEP (2020), Censo Escolar. Disponível em:https://www.gov.br/inep/pt-br/areas-de-atuacao/pesquisas-estatisticas-e-indicadores/censo-escolar/resultados
\end{thebibliography}

\end{document}
